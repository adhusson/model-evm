% !TEX root = main.tex
\def\eval{\mathsf{eval}}
\subsection{Storage and memory allocation}\label{sec:alloc}

\paragraph{Evaluation function.}
We first define the $\eval$ function that maps an expression to a value and an evaluation gas cost:
$$
\begin{array}{llll}
    \eval(\ctx, \textit{gas},r,\memory,\state, op(x)) &=& 
    (op(s.\memory(x)),&\dmr) \\
    \eval(\ctx, \textit{gas},r,\memory,\state, \sload{x_\Uint}) &=& (s.\state(x_\Uint),&\dsr)\\
    \eval(\ctx, \textit{gas}, r,\memory,\state, \mathtt{gasleft}) &=& (s.\textit{gas},&\denv) \\
    \eval(\ctx,\textit{gas}, r,\memory,\state,\mathtt{this}) &=& (\ctx.\thisVal,&\denv) \\
    \eval(\ctx, \textit{gas}, r,\memory,\state, \mathtt{sender}) &=& (\ctx.\senderVal,&\denv) \\
    \eval(\ctx, \textit{gas}, r,\memory,\state,\bcd[i]) &=& (s.\callDataVal[i],&\dcd) \\
    \eval(\ctx, \textit{gas}, r,\memory,\state,\brd[i]) &=& (s.\returnDataVal[i],&\drd) \\
    \eval(\ctx, \textit{gas}, r,\memory,\state, \hsh{x_0,\dots,x_n}) &=& (\mathit{keccak_{256}}(v_0::\dots::v_n), &\dhsh*n)
\end{array}
$$
with $v_i=\memory(x_i)$, and $v::v'$ denoting value concatenation.
 
\paragraph{Storage and memory access derivations.}

$$
\begin{array}{c}
\dfrac{
\eval(\ctx,\textit{gas}, r,\memory,\state,E) = (v,\delta) 
}
{
\ctx\vdash 
(x:=E;\bytecode,\textit{gas}, r, \memory,\state) 
\rar 
(\bytecode,\textit{gas}-\delta-\delta_\mathsf{mw}, r,\memory\set{x\mapsto v},\state)
}\rlab{mstore}
\\\\
\dfrac{
\eval(\ctx,\textit{gas}, r,\memory,\state,E) = (v,\delta)
\quad
\state' = \state\set{\ctx.\thisVal\set{\store\set{\memory(x_{\rm pos})\mapsto v}}}
}
{
\ctx\vdash 
(\sstore{x_{\rm pos},E};\bytecode,\textit{gas}, r, \memory,\state) 
\rar  
(\bytecode,\textit{gas}-\delta-\delta_\mathsf{sw}, r,\memory,\state')
}\rlab{sstore}
\end{array}
$$
