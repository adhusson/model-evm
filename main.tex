% Lower the font size for publication but bigger is easier for previewing
% \documentclass[11pt]{article}
\documentclass[14pt]{extarticle}
\usepackage{amsmath, amssymb, mystyle}
\usepackage{geometry}
% Remove for publication but no margin is easier for previewing
\geometry{margin=.1in}
\begin{document}

\title{A blockchain calculus}
\author{A. Husson, J. Krivine}
\date{}
\maketitle

% opcodes
\def\Op{\mathsf{Op}}
\def\call#1{\mathtt{call}(#1)}
\def\dcall#1{\mathtt{dcall}(#1)}
\def\sstore#1{\mathtt{sstore}(#1)}
\def\sload#1{\mathtt{sload}(#1)}
\def\brd{\mathtt{rdata}}
\def\bcd{\mathtt{cdata}}
\def\hsh#1{\mathtt{hash}(#1)}


\def\return#1{\mathtt{return}(#1)}
\def\revert#1{\mathtt{revert}(#1)}
\def\switch#1{\mathtt{if}(#1)}
\def\while#1{\mathtt{while}(#1)}

% math
\def\upd{\mathrel{\bf\langle\!+\!\!\mid}}
\def\eps{\epsilon}
\def\sig{\sigma}
\def\Nat{\mathbb{N}}
\def\Bool{\mathbb{B}}
\def\Addr{\mathbb{A}}
\def\Str{\mathbb{S}}
\def\tuple#1{\langle{#1}\rangle}
\def\z{\mathbf{0}}
\def\zVec{\langle\rangle}
\def\zVal{[]}


% contracts
\def\Ctr{\mathsf{Ctr}}
\def\Abi{\alpha}
\def\Block{\mathbf{S}}
\def\tx{\mathtt{tx}}

\def\env{\varepsilon}

\def\gasspent{\gamma}
\def\gasmax{\Gamma}
\def\this{\mathop{\downarrow}}
\def\sender{\mathop{\uparrow}}
\def\rd{\varrho}
\def\retData{\vec\rd}
\def\cd{\chi}
\def\callData{\vec\cd}

\def\Byt{\mathtt{B}}

%semantics
\def\Exec{\mathbf{E}}
\def\rlab#1{\;\mathtt{#1}}
\def\llab#1{\mathtt{#1}\;}
\def\comp{\mathrel{\|}}
\def\Pending{\mathsf{Pending}}
\def\Def{\mathsf{Def}}
\def\Require{\mathit{Require}}
\def\Revert{\mathit{Revert}}
\def\rar{\dashrightarrow}

% gas spent
\def\dcl{\delta_{\mathsf{call}}}
\def\dxp{\delta_{\mathsf{exp}}}
\def\dsw{\delta_{\mathsf{sw}}}
\def\dsr{\delta_{\mathsf{sr}}}
\def\dmr{\delta_{\mathsf{mr}}}
\def\denv{\delta_{\mathsf{env}}}
\def\dcd{\delta_{\mathsf{dcd}}}
\def\drd{\delta_{\mathsf{drd}}}

\def\OOG{\mathsf{OOG}}

\section{Terminology}

\subsection{Variables and valuation}
We consider a set of values $V$. In the context of this paper, a type $T$ is subset of $V$ that contains a $\z$ for each type, denoted $\z_T$. We use $\Nat$, addresses $\Addr$, strings $\Str$ and Booleans $\Bool$ as ground types. We also consider the restricted element $\bot\not\in V$.

For all countable set $X$, an \emph{$X$-valuation} is a complete mapping $\nu: X \to V\uplus\set{\bot}$. We use $\Def(\nu):= \set{x\in X\mid \nu(x)\in V}$. We also use $\nu:=[x\mapsto v; y\mapsto v']$ to denote an $X$-valuation $\nu$ such that $\nu(x)=v$, $\nu(y)=v'$ and $\nu(z)=\bot$ for all $z\not\in\set{x,y}$. We let $\nu:=\zVal$, for the valuation that satisfies $\Def(\nu)=\emptyset$. We also write $\nu:=\nu_0\upd \nu_1$ to denote the $X$-valuation satisfying $\nu(x)=\nu_0(x)$ for all $x\not\in\Def(\nu_1)$ and $\nu(x)=\nu_1(x)$ otherwise. In the context of an $X$-valuation $\nu$, we use the naming scheme $x_T$ to denote the fact that $\nu(x_T)\in T$. 

For all $X$-valuation $\nu$ and $\vec x\in X^n$, we write $\nu(\vec x)$ for the vector $\tuple{\nu(x_0),\dots,\nu(x_n)}\in\vec V$.

\subsection{Bytecodes}
A \emph{bytecode} $\Byt$ is a stack of \emph{instructions} $I$, built using the grammar:
$$\begin{array}{lcll}
\Byt &:: =& I ; \Byt \mid \z_\Byt \\
I & :: =& x:=E \mid \sstore{x_\Nat,E} & \hbox{Assignment} \\
&&{}\mid \return {\vec x} \mid \revert x & \hbox{Return and revert} \\
&& {}\mid \switch{x_\Bool, I} \mid \while{x_\Bool,I}  & \hbox{Conditional}\\
&&{}\mid \call{x_\Addr, x_\Str, \vec x, x_\Nat} \mid  \dcall{x_\Addr, x_\Str, \vec x, x_\Nat} & \hbox{Function calls} \\
E & ::= & op(\vec x)\mid \sload{x_\Nat}\mid\hsh{x} & \hbox{Expression} \\ 
&& \mid \mathtt{gasleft} \mid \mathtt{this} \mid \mathtt{sender}\mid \brd_i \mid\bcd_i 
\end{array}
$$
where $op:V^k\to V$ denote any operation on values (constant values are for $k=0$), $\hsh{}$ gives access to a native hash function that maps values to integers, $\mathtt{gasleft}, \mathtt{this}$ and $\mathtt{caller}$ give respectively access to the amount of gas unit left, the address of the execution runner and the address of the last caller of this execution. Each element $\brd_i$ gives access to the \emph{return data} values of the last call and $\bcd_i$ gives access to the \emph{call data} values.

We use $\mathcal{B}$ to denote the set of bytecodes (words recognised by the above grammar).

\paragraph{Example} We can define some basic solidity constructs as macros:
$$ \begin{array} {lll}
\Require(x_\Bool,y) &:=& s_\Bool:=\neg x_\Bool;\switch{s_\Bool,\revert{y}}\\\\
a.f(\vec v,g) &:=& \left\{
\begin{array}{l}
    \vec x:=\vec v; x_\Addr:=a; x_\Str:=f; x_\Nat:=g \\
    \call{x_\Addr,x_\Str,\vec x,x_\Nat}; s_\Bool:=\brd_0; y:=\brd_1\\
    \Require(s_\Bool,y)
\end{array}\right. \\\\
\mathit{try}\ a.f(\vec v,g)\ \mathit{returns}(\vec y)\ \{\Byt\}\ \mathit{catch}\ \set{\Byt'} &:=& \left\{
    \begin{array}{l}
        a.f(\vec v,g); \mathit{success}_\Bool := \brd_0; \mathit{fail}_\Bool := \neg\brd_0\\ 
        \switch{\mathit{success}_\Bool, (y_i:=\brd_{i+1})_{i\leq |\brd|});\Byt} \\
        \switch{\mathit{fail}_\Bool, \Byt'}
    \end{array}\right. \\\\
z:=x_\Nat[y] &:=& y'_\Nat:=\hsh{y}+x_\Nat; z:=\sload{z_\Nat}
\end{array}
$$

\subsection{Smart contracts}
A smart contract $\Ctr$ is given as a pair $$\Ctr:=(\Abi: \Str\to \mathcal{B}\times n, \sigma: \Nat\to V)$$ where $\Abi$ is the contract's \emph{(application binary) interface}, mapping a function name to its bytecode and input arity. The storage $\Nat$-valuation $\sigma$ maps storage positions to values in a persistent manner and we call $\Def(\sig)$ the contract's \emph{(persistent) storage}. We use $\mc C$ to denote the set of possible smart contracts.

A \emph{(blockchain) state} $\Block$ is a mapping $\Block:\Addr\to\mc{C}$. For all contract $\Ctr$, we use $\Ctr.\Abi$ and $\Ctr.\sig$ to denote its first and second projection. Contracts $\Ctr$ such that $\Def(\Ctr.\al)=\emptyset$ are called \emph{externally owned account} (EOA) and must satifsfy $\Def(\Ctr.\sigma)=\emptyset$.

\section{Operational semantics}

\subsection{Execution steps and environment}

An \emph{(execution) environment} is a tuple of the form $\env:=(\gasmax\in\Nat,\gasspent\in\Nat,\this\in\Addr,\sender\in\Addr)$, where $\gasmax$ (\emph{gasmax}) is the max amount of gas an execution step is allowed to spend, $\gasspent$ (\emph{gasspent}) is the gas spent so far, $\this$ (\emph{this}) is the address of the current execution runner, and $\sender$ (\emph{sender}) is the address of the execution's caller. We use $\env.\gasmax, \env.\gasspent, \env.\this$ and $\env.\sender$ to access to $\env$'s fields. For all fields $x,y\in\set{\gasmax,\gasspent,\this,\sender}$, we use the update notation $\env'=\env\set{x= v}$ to denote the environment $\env'$ satisfying $\env'.y = \env.y$ for all $x\neq y$ and $\env.x=v$. 

An \emph{execution step} is a tuple $E:=(\Byt, \callData,\retData, \mu, \Block)$ where $\Byt\in\mc B$ is a bytecode, $\callData\in \vec V$ and $\retData\in\vec V$ are a tuple of \emph{call} and \emph{return values} with $\rd_0\in\Bool$ called the \emph{return status}, $\mu\in V^X$ is a \emph{memory} valuation, and $\Block\in\mc S$ is the blockchain state at that computation step.

An \emph{execution (stack)} $\Exec$, is a stack of execution steps in their environments:
$$
\Exec ::= (\env:E)\cdot\Exec \mid \z_\Exec
$$
We use $\mc E$ to denote the set of possible executions.
\subsection{Transaction initialization}
A \emph{transaction} $t$ is given as a tuple $\tx\in \Addr\times\Str\times V^n\times \Nat\times \Addr$ of the form $\tx:=(a,f,\vec v, G, o)$ where $a$ is the address of the called contract, $f$ a function name and $\vec v$ its arguments. The maximum amount of gas the transaction can spend is given by $G$ and $o$ is the transaction originator. For any transaction $\tx$ we use 
$\tx.a, \tx.f, \tx.{\vec v},\tx.G,\tx.o$ to access a particular field.
We use $\mc T$ to denote the set of transactions.

Define the $\mathsf{init}: \mc T\times \mc S \to \mc E$ as:
%%
$$
\mathsf{init}(\tx, \Block) :=  \env:(\call{x_\Addr,x_\Str,\vec x, x_\Nat};\z_\Byt,\zVec,\zVec,\mu,\Block)\cdot \z_\Exec
$$ 
with:
$$
\begin{array}{cc}
\mu := \left[ 
    \begin{array}{lll}
    x_\Addr & \mapsto & \tx.a \\
    x_\Str & \mapsto & \tx.f \\
    x_i &\mapsto& \tx.v_i \\
    x_\Nat & \mapsto & \tx.G \\
    \end{array}\right] 
& \hbox{ and }
\left\{
\begin{array}{lll}
    \env.\gasmax &:=&\tx.G \\
    \env.\gasspent &:=& 0 \\
    \env.\this&:=&\tx.o \\ 
    \env.\sender&:=&\tx.o 
\end{array}
\right.
\end{array}
$$

\subsection{Checking gas limit}
We define the following \emph{Out of Gas} predicate:
$$
\OOG(\env) := \env.\gasspent > \env.\gasmax 
$$
It is used to derive the generic derivations:
\begin{equation}
\dfrac{
\env:E\cdot\Exec\rar \env':E'\cdot\Exec' \quad \OOG(\env')
}{
\env:E\cdot\Exec\to \env: (\revert{x},\zVec,\zVec,\zVal,\z_\Block)\cdot\Exec
}
\end{equation}

\begin{equation}
\dfrac{
\env:E\cdot\Exec\rar \env':E'\cdot\Exec' \quad \neg\OOG(\env')
}{
\env:E\cdot\Exec\to \env':E'\cdot\Exec'
}
\end{equation}

\newpage 
\subsection{Function calls derivations}
\def\callable#1{#1\hbox{\--}\mathsf{Callable}}
Let $n \in\Nat$, $a\in\Addr$ and $f\in\Str$. We define the following predicate:
$$
\callable{\Byt}(a,f,n,\Block) := \exists \sigma\in V^X.\Block(a)=(\Abi,\sigma) \wedge \Abi(f)= (\Byt,n) 
$$

\begin{table}[ht]
$$
\begin{array}{c}
\dfrac{
    \begin{array}{lll}
        \env'.\gasmax &=& \mu(x_\Nat) \\
        \env'.\gasspent &=& \env.\gasspent + \dcl \\
        \env'.\this & = & \mu(x_\Addr) \\
        \env'.\sender & = & \env.\this
    \end{array}
    \quad
    \begin{array}{l}
        \mu(x_\Nat) \leq \env.\gasmax \\
        \callable{\Byt'}(\mu(x_\Addr),\mu(x_\Str),|\vec x|,\Block)
    \end{array}
}
{
 \env:(\call{x_\Addr,x_\Str,\vec x, x_\Nat}\cdot\Byt,\callData, \retData, \mu, \Block)\cdot\Exec 
 \rar 
 \env':(\Byt',\mu(\vec x), \zVec, \zVal,\Block)\cdot\env:(\Byt, \callData, \retData, \mu, \Block)\cdot\Exec
}\rlab{call}
\\\\
\dfrac{
    \begin{array}{lll}
        \env'.\gasmax &=& \mu(x_\Nat) \\
        \env'.\gasspent &=& \env.\gasspent + \dcl \\
        \env'.\this & = & \env.\this \\
        \env'.\sender & = & \env.\sender
    \end{array}
    \quad 
    \begin{array}{l}
        \mu(x_\Nat) \leq \env.\gasmax \\
        \callable{\Byt'}(\mu(x_\Addr),\mu(x_\Str),|\vec x|, \Block)
    \end{array} 
}
{
 \env:(\dcall{x_\Addr,x_\Str,\vec x, x_\Nat}\cdot\Byt, \callData, \retData, \mu, \Block)\cdot\Exec 
 \rar 
 \env':(\Byt', \mu(\vec x), \zVec, \zVal,\Block)\cdot\env:(\Byt, \callData, \retData,\mu, \Block)\cdot\Exec
}\rlab{dcall}
\\\\
\dfrac{
\env' = \env\set{\gasspent = (\env.\gasspent + \dcl)}
\quad 
\mu(x_\Nat) \leq \env.\gasmax \quad
\forall\Byt'.\neg\callable{\Byt'}(\mu(x_\Addr),\mu(x_\Str),|\vec x|,\Block) 
}
{
 \env:(\mathsf{[d]call}(x_\Addr, x_\Str, \vec x, x_\Nat)\cdot\Byt, \callData,\retData,\mu, \Block)\cdot\Exec 
 \rar 
 \env': (\revert{x}, \zVec,\zVec,\zVal, \z_\Block)\cdot\Exec
}\rlab{[d]callFail}

\end{array}
$$
\caption{Call and delegate call derivations}
\end{table}

\newpage
\subsection{Return and revert derivations} 

\begin{figure}[ht]
$$
\begin{array}{c}
\dfrac{\env''=\env\set{\gasspent = \env'.\gasspent}\quad \retData''=[\mathbf{1}_\Bool]}
{
\env':(\z_\Byt, \callData', \retData', \mu', \Block')\cdot\env:(\Byt,\callData,\retData,\mu,\Block)\cdot\Exec \to \env'':
(\Byt,\callData,\retData'',\mu,\Block')\cdot\Exec}\rlab{impl\_return} 
\\\\
\dfrac{\env''=\env\set{\gasspent = \env'.\gasspent}\quad \retData''=[\mathbf{1}_\Bool]\upd\mu'(\vec x)}
{
\env':(\return{\vec x};\Byt', \callData', \retData', \mu', \Block')\cdot\env:(\Byt,\callData,\retData,\mu,\Block)\cdot\Exec \to
\env'':(\Byt,\callData,\retData'',\mu,\Block')\cdot\Exec}\rlab{expl\_return} 
\\\\
\dfrac{\env''=\env\set{\gasspent = \env'.\gasspent}\quad \retData''=[\mathbf{0}_\Bool;\mu'(x)]}
{
\env':(\revert{x};\Byt', \callData', \retData', \mu', \Block')\cdot\env:(\Byt,\callData,\retData,\mu,\Block) \cdot\Exec 
\to 
\env'':(\Byt,\callData,\retData'',\mu,\Block)  \cdot\Exec}\rlab{revert}
\end{array}
$$
\caption{Return and revert derivations for transaction executions.}
\end{figure}

\newpage
\def\eval{\mathsf{eval}}
\subsection{Storage and memory allocation}
We first define the $\eval$ function that maps an expression to value and a gas cost:
$$
\begin{array}{llll}
    \eval(\ctx, \textit{gas},r,\memory,\state, op(x)) &=& 
    (op(s.\memory(x)),&\dmr) \\
    \eval(\ctx, \textit{gas},r,\memory,\state, \sload{x_\Nat}) &=& (s.\state(x_\Nat),&\dsr)\\
    \eval(\ctx, \textit{gas}, r,\memory,\state, \mathtt{gasleft}) &=& (s.\textit{gas},&\denv) \\
    \eval(\ctx,\textit{gas}, r,\memory,\state,\mathtt{this}) &=& (\ctx.\thisVal,&\denv) \\
    \eval(\ctx, \textit{gas}, r,\memory,\state, \mathtt{sender}) &=& (\ctx.\senderVal,&\denv) \\
    \eval(\ctx, \textit{gas}, r,\memory,\state,\bcd[i]) &=& (s.\callDataVal[i],&\dcd) \\
    \eval(\ctx, \textit{gas}, r,\memory,\state,\brd[i]) &=& (s.\returnDataVal[i],&\drd)
\end{array}
$$
\begin{figure}[ht]
$$
\begin{array}{c}
\dfrac{
\eval(\ctx,\textit{gas}, r,\memory,\state,E) = (v,\delta) 
}
{
\ctx\vdash 
(x:=E;\bytecode,\textit{gas}, r, \memory,\state) 
\rar 
(\bytecode,\textit{gas}-\delta-\delta_\mathsf{mw}, r,\memory\set{x\mapsto v},\state)
}\rlab{mstore}
\\\\
\dfrac{
\eval(\ctx,\textit{gas}, r,\memory,\state,E) = (v,\delta)
\quad
\state' = \state\set{\ctx.\thisVal\set{\store\set{\memory(x_\Nat)\mapsto v}}}
}
{
\ctx\vdash 
(\sstore{x_\Nat,E};\bytecode,\textit{gas}, r, \memory,\state) 
\rar 
(\bytecode,\textit{gas}-\delta-\delta_\mathsf{sw}, r,\memory,\state')
}\rlab{sstore}
\end{array}
$$
\caption{Allocation rules.}
\end{figure}
\newpage
\subsection{Conditionals and loops}

\begin{figure}[ht]
$$
\begin{array}{c}
\dfrac{
\env'=\env\set{\gasspent=\env.\gasspent+\dmr} \quad
\Byt'' = \left\{\begin{array}{ll}
\Byt';\Byt &\hbox{if}\ \mu(x_\Bool) \\
\Byt &\hbox{else}
\end{array}\right.
} 
{
\env:(\switch{x_\Bool,\Byt'};\Byt,\callData,\retData, \mu, \Block)\cdot\Exec \rar\env':(\Byt'',\callData, \retData, \mu, \Block)\cdot\Exec
}\rlab{if}
\\\\
\dfrac{
\env'=\env\set{\gasspent=\env.\gasspent+\dmr} 
\quad \Byt'' = \left\{\begin{array}{ll}
\Byt';\while{x_\Bool,\Byt'};\Byt &\hbox{if}\ \mu(x_\Bool) \\
\Byt &\hbox{else}
\end{array}\right.} 
{
\env:(\while{x_\Bool,\Byt'};\Byt,\callData,\retData,\mu, \Block)\cdot\Exec \rar\env':(\Byt'', \callData,\retData, \mu, \Block)\cdot\Exec
}\rlab{while} 
\end{array}
$$
\caption{Conditional and while loop.}
\end{figure}


\section{Black box interaction}

\subsection{Bytecode abstraction}
\def\aByt{{\Byt^\sharp}}
\def\burn{\dagger}

We extend the bytecode grammar:
$$\begin{array}{lcll}
\aByt &:: =& \Byt \mid \aByt\vee \aByt
\end{array}
$$
with the following (non deterministic) interpretation:
$$
\begin{array}{c}
\dfrac{\env:(\Byt,\callData,\retData,\mu,\Block)\cdot\Exec\rar \Exec'}
{\env:(\Byt\vee \Byt',\callData,\retData,\mu,\Block)\cdot\Exec\rar \Exec'}
\qquad
\dfrac{\env:(\Byt,\callData,\retData,\mu,\Block)\cdot\Exec\rar \Exec'}
{\env:(\Byt'\vee \Byt,\callData,\retData,\mu,\Block)\cdot\Exec\rar \Exec'}
\end{array}
$$

\subsection{Abstract custodian contracts}


\end{document}
