% Lower the font size for publication but bigger is easier for previewing
% \documentclass[11pt]{article}
\documentclass[14pt]{extarticle}
\usepackage{amsmath, amssymb, mystyle}
\usepackage{geometry}
\usepackage{cancel}
\usepackage{xspace}
% Remove for publication but no margin is easier for previewing
\geometry{margin=.1in}
\begin{document}

\title{A blockchain calculus}
\author{A. Husson, J. Krivine}
\date{}
\maketitle

% opcodes
\def\Op{\mathsf{Op}}
\def\call#1{\mathtt{call}(#1)}
\def\dcall#1{\mathtt{dcall}(#1)}
\def\sstore#1{\mathtt{sstore}(#1)}
\def\sload#1{\mathtt{sload}(#1)}
\def\brd{\mathtt{rdata}}
\def\bcd{\mathtt{cdata}}
\def\hsh#1{\mathtt{hash}(#1)}
\def\return#1{\mathtt{return}(#1)}
\def\revert#1{\mathtt{revert}(#1)}
\def\switch#1{\mathtt{if}(#1)}
\def\while#1{\mathtt{while}(#1)}

% math
\def\Nat{\mathbb{N}}
\def\Uint{\mathbb{U}}
\def\Bool{\mathbb{B}}
\def\Addr{\mathbb{A}}
\def\Str{\mathbb{S}}
\def\ary#1{\left[{#1}\right]}
\def\z{\mathbf{0}}
\def\zMem{\z_\memory}
\def\zArray{[]}
\def\zStore{\z_\textit{store}}
\def\zBytecode{\z_{\texttt{B}}}


% contracts
\def\Ctr{\mathsf{Ctr}}
\def\Abi{\textit{ABI}}
\def\Block{\mathbf{S}}
\def\tx{\mathsf{tx}}
\def\transactions{\textit{Transactions}}
\def\ctx{\mathsf{Ctx}}

\def\thisVar{\texttt{this}}
\def\thisVal{\textit{this}}

\def\stateVal{\textit{state}}

\newcommand{\initStep}[3]{\mathsf{Step}^{#1}_{#2.#3}}

\def\senderVar{\texttt{sender}}
\def\senderVal{\textit{sender}}

\def\returnDataVal{\textit{returndata}}
\def\dataVal{\textit{data}}
\def\returnDataVar{\texttt{returndata}}
\def\gasleftVal{\textit{gasleft}}
\def\gas{\textit{gas}}
\def\gasleftVar{\texttt{gasleft}}
\def\originVal{\textit{origin}}
\def\toVal{\textit{to}}
\def\fnVal{\textit{fn}}
\def\gasVal{\textit{gas}}
\def\cd{\chi}
\def\callDataVar{\texttt{calldata}}
\def\callDataVal{\textit{calldata}}
\def\memory{\textit{mem}}
\def\step{\mathsf{Step}}
\def\steps{\textit{Steps}}
\def\endStep{\mathsf{End}}
\def\continuation#1{\textit{continuation}(#1)}

\def\Byt{\mathtt{B}}

%semantics
\def\Exec{\mathbf{E}}
\def\rlab#1{\;\mathtt{#1}}
\def\llab#1{\mathtt{#1}\;}
\def\comp{\mathrel{\|}}
\def\Pending{\mathsf{Pending}}
\def\Def{\mathsf{Def}}
\def\require{\texttt{require}}
\def\Revert{\mathit{Revert}}
\def\rar{\dashrightarrow}

% gas spent
\def\dcl{\delta_{\mathsf{call}}}
\def\dxp{\delta_{\mathsf{exp}}}
\def\dsw{\delta_{\mathsf{sw}}}
\def\dsr{\delta_{\mathsf{sr}}}
\def\dmr{\delta_{\mathsf{mr}}}
\def\denv{\delta_{\mathsf{env}}}
\def\dcd{\delta_{\mathsf{dcd}}}
\def\drd{\delta_{\mathsf{drd}}}

\def\OOG{\mathsf{OOG}}

\def\vars{\ensuremath{\textit{Vars}}}
\def\vals{\ensuremath{\textit{Vals}}}
\def\valueOf{\textit{valueOf}}
\def\emptyVal{\textit{emptyValuation}}
\def\falseVal{\textit{false}}
\def\trueVal{\textit{true}}
\def\store{\textit{store}}

\def\bytecodes{\textit{Bytecodes}}
\def\bytecode{\textit{B}}
\def\code{\textit{code}}
\def\contracts{\textit{Contracts}}
\def\state{\textit{state}}
\def\states{\textit{States}}

\section{Terminology2}

\subsection{Terminology}
\paragraph{Values and variables}
We consider a finite set of values $\vals$ with a distinguished zero value $\z$. All common types (strings, numbers, booleans, addresses) are assumed to be transparently encoded to and from this value set (so for instance, any string is a value and any value is a string).

For convenience, we can talk about the set of strings $\Str$, of addresses $\Addr$, of unsigned integers $\Uint$, or of booleans $\Bool$. But we are really talking about the set of values.

Common operations such as $+$, $-$, comparison, etc are defined on values.

\paragraph{Variables}
We use variables, or names, as field names in the smart contract language, and also in the language semantics. There is a countable set of names $\vars$. 

\paragraph{Memory}
Contracts use memory as a transient scratchpad.
% We use the following ground types: $\Nat$ for naturals, $\Addr$ for account addresses, $\Str$ for strings and $\Bool$ for booleans. We also consider the restricted element $\bot\not\in\vals$.
A \emph{memory} or $\memory$ is a total mapping from names to values. The memory $(x\mapsto v,y\mapsto v')$ maps $x$ to $v$, $y$ to $v'$, and maps every other name to $\z$. $\zMem$ is the empty memory, it maps every name to $\z$.

\paragraph{Tuples}
We use tuples everywhere to structure definitions. A \emph{tuple} is a partial mapping from names (to anything). $()$ is the empty tuple.

\subparagraph{Example} 
We will later introduce \emph{contracts} as tuples of the form $(\Abi\in\Str\rightarrow\bytecodes,\store\in\Nat\rightarrow\vals)$. 
This means that all contracts have domain $\set{\Abi,\store}\subseteq\vars$, that they all map $\Abi$ to strings-to-bytecode mappings, and that they all map $\store$ to integers-to-value mappings.
Given concrete values $A$ and $S$, $C = (\Abi\mapsto A,\store\mapsto D)$ is the contract such that $C(\Abi) = A$ and $C(\store) = D$

\subparagraph{}
When convenient the name of a tuple member is omitted and its position implicitly denotes the corresponding name. 
\subparagraph{Example}
$C' = (A',D')$ is the contract with domain $\set{\Abi,\store}$ such that $C'(\Abi) = A'$ and $C'(\store) = D'$

\paragraph{Stores} Contracts use their store (or storage) as long-term memory. A $\store$ is a total mapping from values (representing integers) to values. The store $(2\mapsto v)$ maps $2$ to $v$, and it maps every other integer to $\z$. $\zStore$ is the empty store, it maps every number to $\z$.

\paragraph{Arrays}
\emph{Arrays} are partial mappings from integers (to anything) with no hole in the domain of definition. $[]$ is the empty array (aka the function defined nowhere). For any set $X$, the set of arrays with all elements in $X$ is $\vec X$.
\subparagraph{Examples}
$a = [3,0,1]$ has domain $\mathbf{3}$. $a(0) = 3$, $a(1) = 0$, $a(2) = 1$.

\def\Dom{\textrm{Dom}}
\paragraph{Update syntax}
A mapping can be defined as an update of an existing mapping. If a mapping $m$ is defined on $x$ then $m' = m\set{x\mapsto v}$ is equal to $m$, except that $m'(x) =v$. The notation is extended to more than one element.

\subparagraph{Example} $C\set{\store\mapsto D'}$ is the contract $(\Abi\mapsto A,\store\mapsto D')$. $a\set{1\mapsto10}$ is the array $[3,10,1]$.

\subparagraph{Nested update syntax} If $t$ is a tuple and $t.f$ is a mapping, then $t\set{f\set{\dots}} = t\set{f\mapsto t.f\set{\dots}}$.

\subparagraph{Example} If $\state$ is a tuple, $\state.\textit{address}$ is a tuple, and $\state.\textit{address}.\store$ is a store, then $\state\set{\textit{address}\set{\store\set{3 \mapsto 7}}}$ is equal to $\state$ except that $\state.\textit{address}.\store(3) = 7$.

\subsection{Bytecodes}
A \emph{bytecode} $\bytecode$ is a stack of \emph{instructions} $I$, built using the grammar (for all $n$):
$$\begin{array}{lcll}
\bytecode &:: =& I ; \bytecode \mid \z \\
I & :: =& x:=E \mid \sstore{x_\textrm{loc},E} & \hbox{Assignment} \\
&&{}\mid \return {x_0,\dots,x_n} \mid \revert{x_0,\dots,x_n} & \hbox{Return and revert} \\
&& {}\mid \switch{x_\textrm{cond}, B} \mid \while{x_\textrm{cond},B}  & \hbox{Conditional}\\
&&{}\mid \call{x_\textrm{to}, x_\textrm{fn}, x_0, \dots, x_n,x_\textrm{gas}}\\
&&{}\mid \dcall{x_\textrm{to}, x_\textrm{fn}, x_0, \dots, x_n,x_\textrm{gas}}
 & \hbox{Function calls} \\
E & ::= & op(x_0,\dots,x_n)\mid \sload{x_\textrm{loc}}\mid\hsh{x} & \hbox{Expression} \\ 
&& \mid \gasleftVar \mid \thisVar \mid \mathtt{sender}\mid \brd[i] \mid\bcd[i] 
\end{array}
$$
where $op:\vals^n\to \vals$ denote any operation on values (constant values are for $n=0$), $\hsh{}$ gives access to a native hash function that maps values to integers, 
$\gasleftVar$, $\thisVar$ and $\senderVar$ give respectively access to the amount of gas unit left, the address of the execution runner and the address of the last caller of this execution. 
Each element $\brd[i]$ gives access to the \emph{return data} values of the last call and 
$\bcd[i]$ gives access to the \emph{call data} values.

We use $\bytecodes$ to denote the set of bytecodes (words recognised by the above grammar).

% \paragraph{Bytecode continuation} If the bytecode $\bytecode$ is of the form $I;\bytecode'$, then $\continuation{\bytecode} := \bytecode'$.


\paragraph{Example} We can define some basic solidity constructs as macros (introduced variables are fresh):
$$ \begin{array} {lll}
\switch{E,B} &:=& x_\textrm{cond} := E; \switch{x_\textrm{cond},B}\\\\
\dots &:=& \dots\\\\
\dcall{E_\textrm{to},E_{\textrm{fn}},E_0,\dots,E_n,E_{\textrm{gas}}} &:=& \left\{
\begin{array}{l}
x_\textrm{to} := E_\textrm{to};x_\textrm{fn} := E_\textrm{fn}; \\
x_0 := E_0; \dots; x_n := E_n; \\
x_\textrm{gas} := E_\textrm{gas}\\
\dcall{x_\textrm{to},x_\textrm{fn},x_0,\dots,x_n,x_\textrm{gas}}\\\end{array}\right.\\\\
\require(E_\textrm{cond},E_\textrm{reason}) &:=& \switch{\neg E_\textrm{cond},\revert{E_\textrm{reason}}}\\\\
\texttt{$a$.$f$\{gas:$E_\textrm{gas}$\}($E_0,\dots,E_n$)} &:=& \left\{
\begin{array}{l}
    \call{a,f,E_0,\dots,E_n,E_\textrm{gas}}; \\
    \require(\brd[0],\brd[1])
\end{array}\right. \\\\
\begin{array}{l}
\texttt{try $a$.$f$\{gas:$E_\textrm{gas}$\}($E_0,\dots,E_n$)}\\
\texttt{returns($x_0,\dots,x_m$)}\\
\texttt{\{\bytecode\} catch \{\bytecode'\}}\end{array} &:=& \left\{
    \begin{array}{l}
        \call{a,f,\vec v,g}; \mathit{success}_\Bool := \brd_0\\
        \switch{\mathit{success}_\Bool, (y_i:=\brd_{i+1})_{i\leq |\brd|});\bytecode} \\
        \switch{\neg\mathit{success}_\Bool, \bytecode'}
    \end{array}\right. \\\\
z:=x_\Nat[y] &:=& y'_\Nat:=\hsh{y}+x_\Nat; z:=\sload{z_\Nat}
\end{array}
$$

\subsection{Smart contracts}
A smart contract $\Ctr$ is a tuple of the form $$\Ctr:=(\Abi\in \Str\to \bytecodes, \store\in \Nat\to \vals)$$ where $\Abi$ is the contract's \emph{(application binary) interface}, mapping a function name to its bytecode. The contract storage maps positions to values in a persistent manner. We use $\contracts$ to denote the set of possible smart contracts.

A \emph{(blockchain) state} is a mapping $\state\in \Addr\to\contracts$. The set of all possible states is $\states$.

\section{Operational semantics}

\subsection{Execution steps and environment}

An \emph{(execution) context} is a tuple of the form $\ctx:=(\callDataVal\in\vec\vals,\thisVal\in\Addr,\senderVal\in\Addr)$, 
where $\callDataVal$ is an array of values given by the caller, $\thisVal$ is the address of the current execution runner, 
and $\senderVal$ is the address of the execution's caller. 

An \emph{(execution) step} is a tuple of the form
$\step:=(
    \code\in\bytecodes,
    \gas\in\Uint,
    \returnDataVal\in\vec\vals, 
    \memory\in\vars\to\vals, 
    \state\in\states
)$. The set of all steps is $\steps$.

\subsection{Transaction initialization}
A \emph{transaction} $t$ is given as a tuple of the form $\tx:=(\originVal\in\Addr,\toVal\in\Addr,\fnVal\in\Str,\callDataVal\in\vec \vals, \gasVal\in\Uint)$ where $\originVal$ is the transaction originator, $\toVal$ is the address of the called contract, $\fnVal$ a function name and $\vec c$ its call data. The maximum amount of gas the transaction can spend is given by $\gasVal$.
We use $\transactions$ to denote the set of transactions.

A transaction $\tx$ executed on a blockchain $\state$ induces an initial context $\ctx_\tx$ and an initial step $\step_\tx$:

$$ \begin{array} {lll}
\ctx(\tx) &:=& (\tx.\originVal,\tx.\toVal,\tx.\callDataVal)\\\\
\step(\tx)&:=& \state \mapsto (\state(\tx.\toVal).\Abi(\tx.\fnVal),\tx.\gas, \zArray,\zMem,\state)\\\\
\end{array}$$



\subsection{Checking gas limit}
An execution step is out of gas if it has less than $0$ gas left. Derivation rules $\rar$ below don't check gas, but the steps are only valid ($\rightarrow$) if the step has some gas left. In the derivation below, \texttt{"out of gas"} is a value that represents the string \emph{out of gas}.
\begin{equation}
\dfrac{
\ctx\vdash\step\rar \step' \quad \step'.\gas < 0
}{
\ctx\vdash\step\to \step\set{\bytecode\mapsto \revert{\texttt{"out of gas"}};\zBytecode}
}
\end{equation}

\begin{equation}
\dfrac{
\ctx\vdash\step\rar \step' \quad \step'.\gas \geq 0
}{
\ctx\vdash\step\to \step'
}
\end{equation}

\newpage 
\subsection{Function calls derivations}
\subsubsection{Final steps}
A step $s$ that does not rewrite to any step through $\rightarrow$ is \emph{final}. We denote it by $s \nrightarrow$. 
\paragraph{Returning steps}

A step $s$ \emph{returns} an array $r$ when either:
\begin{itemize}
    \item $r = [\trueVal,s.\memory(x_1),\dots,s.\memory(x_n)]$ if $s.\code = \return{x_1,\dots,x_n};\bytecode$, for some $x_i$\,s and $\bytecode$, and
    \item $r = [\trueVal]$ if $s.\code = \zBytecode$
\end{itemize}
\subparagraph{Exercise} Show that any returning step is final (hint: check derivation rules).
 


\paragraph{Reverting steps}
Any non-returning final step $s$ reverts $r$, where
\begin{itemize}
    \item $r = \ary{\falseVal, s.\memory(x_1),\dots, s.\memory(x_n)}$ if $s.\code = \revert{x_1,\dots,x_n};\bytecode$, for some $x_i$s and $\bytecode$, and
    \item $r = \ary{\falseVal}$ otherwise
\end{itemize}

\subsubsection{Context nesting}
% We say that \emph{$\ctx\vdash s$ runs $\bytecode'$ with gas $G$ under $\ctx'$ and continuation $\bytecode$} when $s.\bytecode = B';\bytecode$ and:

We say that \emph{$\ctx\vdash s$ creates $\ctx'\vdash(B',G')| B$} when

$$
\begin{array}{ll}
&\left\{
    \begin{array}{lll}
    s.\code &=&  \call{x_\Addr,x_\Str,x_1,\dots,x_n, x_\Nat};B \\
    \ctx' &=&(\callDataVal\mapsto c,
     \thisVal\mapsto a,
     \senderVal\mapsto\ctx.\thisVal)
    \end{array}\right.\\
\hbox{ or }&\\
&\left\{
    \begin{array}{lll}
    s.\code &=& \dcall{x_\Addr,x_\Str,x_1,\dots,x_n, x_\Nat};B \\
    \ctx'&=&\ctx\set{\callDataVal\mapsto c},
    \end{array}
    \right.
\end{array}
$$
$$
\begin{array}{cc}
\hbox{and }
\left\{
    \begin{array}{lll}
    \memory(x_\Addr) & = & a \\
    \memory(x_\Str) & = & f \\
    \memory(x_i) & = & c[i]\quad \forall i\leq n \\
    \memory(x_\Nat) & = & G' \\
    \bytecode{}' &=& s.\state(a).\Abi(f)\\
    s.\gas &\geq& G' + \dcl
    \end{array}\right.
\end{array}
$$
\subsubsection{Derivation of a function call}

\begin{table}[ht]
$$
\begin{array}{c}
\dfrac{
    \begin{array}{lll}
    % \multicolumn{3}{l}{\textrm{$\ctx\vdash s$ runs $\bytecode'$ with gas $G$ under $\ctx'$ and continuation $\bytecode$}}\\
    \multicolumn{3}{l}{\textrm{$\ctx\vdash s$ creates $\ctx'\vdash (B',G')|B$}}\\
     \mathsf{\ctx'}&\vdash& 
     (\bytecode{}{}',G,[],\zMem,s.\state) \to^* t \nrightarrow\\
     \state &:=& \left\{
    \begin{array}{l}
        \textrm{$t.\state$ if $t$ returns $r$}\\
        \textrm{$s.\state$ if $t$ reverts $r$}
    \end{array}\right.\\
    \textit{gas} &:=& s.\textit{gas}-\dcl+t.\textit{gas}-G'
    \end{array}
}{
 \ctx
 \vdash
 s
 \rar 
 (\bytecode,\textit{gas},r,s.\textit{mem},
 \state
 )
}\rlab{xcall}
\end{array}
$$
\end{table}


\newpage
\def\eval{\mathsf{eval}}
\subsection{Storage and memory allocation}
We first define the $\eval$ function that maps an expression to value and a gas cost:
$$
\begin{array}{llll}
    \eval(\ctx, \textit{gas},r,\memory,\state, op(x)) &=& 
    (op(s.\memory(x)),&\dmr) \\
    \eval(\ctx, \textit{gas},r,\memory,\state, \sload{x_\Nat}) &=& (s.\state(x_\Nat),&\dsr)\\
    \eval(\ctx, \textit{gas}, r,\memory,\state, \mathtt{gasleft}) &=& (s.\textit{gas},&\denv) \\
    \eval(\ctx,\textit{gas}, r,\memory,\state,\mathtt{this}) &=& (\ctx.\thisVal,&\denv) \\
    \eval(\ctx, \textit{gas}, r,\memory,\state, \mathtt{sender}) &=& (\ctx.\senderVal,&\denv) \\
    \eval(\ctx, \textit{gas}, r,\memory,\state,\bcd[i]) &=& (s.\callDataVal[i],&\dcd) \\
    \eval(\ctx, \textit{gas}, r,\memory,\state,\brd[i]) &=& (s.\returnDataVal[i],&\drd)
\end{array}
$$
\begin{figure}[ht]
$$
\begin{array}{c}
\dfrac{
\eval(\ctx,\textit{gas}, r,\memory,\state,E) = (v,\delta) 
}
{
\ctx\vdash 
(x:=E;\bytecode,\textit{gas}, r, \memory,\state) 
\rar 
(\bytecode,\textit{gas}-\delta-\delta_\mathsf{mw}, r,\memory\set{x\mapsto v},\state)
}\rlab{mstore}
\end{array}

\\\\
\begin{array}{c}
\dfrac{
\eval(\ctx,\textit{gas}, r,\memory,\state,E) = (v,\delta)
\quad
\state' = \state\set{\ctx.\thisVal\set{\store\set{\memory(x_\Nat)\mapsto v}}}
}
{
\ctx\vdash 
(\sstore{x_\Nat,E};\bytecode,\textit{gas}, r, \memory,\state) 
\rar 
(\bytecode,\textit{gas}-\delta-\delta_\mathsf{sw}, r,\memory,\state')
}\rlab{sstore}
\end{array}
$$
\caption{Allocation rules.}
\end{figure}
\newpage
\subsection{Conditionals and loops}

\begin{figure}[ht]
$$
\begin{array}{c}
\dfrac{
s.\code =  \switch{x_\textrm{cond},\bytecode'};\bytecode
\quad \bytecode'' = \left\{\begin{array}{ll}
\bytecode  &\hbox{if}\ s.\memory(x_\textrm{cond}) = \z \\
\bytecode';\bytecode&\hbox{otherwise} \\
\end{array}\right.
} 
{
\ctx \vdash s \rar s\set{
\gas\mapsto g-\dmr,
\code\mapsto \bytecode''}
}\rlab{if}
\\\\
\dfrac{
s.\code =  \switch{x_\textrm{cond},\bytecode'};\bytecode
\quad \bytecode'' = \left\{\begin{array}{ll}
\bytecode &\hbox{if}\ s.\memory(x_\textrm{cond}) = \z\\
\bytecode';\while{x_\textrm{cond},\bytecode'};\bytecode &\hbox{otherwise}
\end{array}\right.
} 
{
\ctx \vdash s \rar s\set{\gas\mapsto g-\dmr,\code\mapsto \bytecode''}
}\rlab{while} 
\end{array}
$$
\caption{Conditional and while loop.}
\end{figure}


\subsection{Abstract custodian contracts}


\end{document}
